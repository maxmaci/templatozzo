% SVN info for this file
\svnidlong
{$HeadURL$}
{$LastChangedDate$}
{$LastChangedRevision$}
{$LastChangedBy$}

\chapter{Comandi personalizzati}
\labelChapter{CustomCommands}
\begin{introduction}
	``...''
	\begin{flushright}
		\textscsl{John Milnor}
	\end{flushright}
\end{introduction}
\lettrine[findent=1pt, nindent=0pt]{I}{n questo} capitolo sono elencati tutti i comandi personalizzati e i loro output, con alcuni commenti addizionali sul loro Modalità d'uso, il loro scopo ed altre osservazioni.
\section{Comandi generici per \LaTeX}
\paragraph{\textbackslash underbfsf}
\begin{equation*}
	\underbfsf{testo}
\end{equation*}
\textbf{\textit{Descrizione:}} testo in grassetto, sans serif e sottolineato.\\
\textbf{\textit{Modalità d'uso:}}
\begin{itemize}
	\item \inlinecode{\#1} è il testo.
\end{itemize}
\begin{codelatex}
\underbfsf{#1}
\end{codelatex}
\begin{example}{}
	La categoria $\underbfsf{Set}$ ha come oggetti gli insiemi: se sapete l'inglese, è un fatto \underbfsf{ovvio}.
\end{example}
\begin{codelatex}
La categoria $\underbfsf{Set}$ ha come oggetti gli insiemi: se sapete l'inglese, è un fatto \underbfsf{ovvio}.
\end{codelatex}
\paragraph{\textbackslash Dfrac}
\begin{equation*}
	\Dfrac{numeratore}{denominatore}
\end{equation*}
\textbf{\textit{Descrizione:}} frazione con numeratore e denominatore in \inlinecode{\textbackslash displaystyle}.\\
\textbf{\textit{Modalità d'uso:}} ambiente matematico;
\begin{itemize}
	\item \inlinecode{\#1} è il numeratore;
	\item \inlinecode{\#2} è il denominatore;
\end{itemize}
\begin{codelatex}
\Dfrac{#1}{#2}
\end{codelatex}
\begin{example}{}
	Data una $n$-upla di elementi $(x_i)_i$ con pesi associati (non negativi) $(w_i)_i$, la media ponderata è \begin{equation*}
		\overline{x}=\Dfrac{\sum_{i=1}^{n}w_ix_i}{\sum_{i=1}^{n}w_i}.
	\end{equation*}
\end{example}
\begin{codelatex}
Data una $n$-upla di elementi $(x_i)_i$ con pesi associati (non negativi) $(w_i)_i$, la media ponderata è
\begin{equation*}
	\overline{x}=\Dfrac{\sum_{i=1}^{n}w_ix_i}{\sum_{i=1}^{n}w_i}.
\end{equation*}
\end{codelatex}
\section{Comandi utili per dimostrazioni}
\paragraph{\textbackslash rightimplies, \textbackslash leftimplies}
\begin{equation*}
	\rightimplies\qquad\qquad\leftimplies
\end{equation*}
\textbf{\textit{Descrizione:}} un'implicazione destra e sinistra con parentesi tonda e spazio incluso.\\
\textbf{\textit{Modalità d'uso:}} nessun argomento aggiuntivo.\\
\textbf{\textit{Scopo:}} per indicare quale direzione di una doppia implicazione $(\iff)$ si sta dimostrando.\\
\textbf{\textit{Osservazioni:}} per motivi a me al momento sconosciuti, se si usano questi comandi nella prima riga degli environment senza l'opzione \inlinecode{nonewline} c'è una piccola e fastidiosa indentazione; al momento, per risolvere tale bug è necessario passare l'opzione \inlinecode{n} (\textit{no new line}) e indurre manualmente un \textit{a capo} iniziale con \inlinecode{$\sim$\{\}\textbackslash\textbackslash}.%TODO: correggere il bug, creare il comando per ~{}\\.
\begin{example}{}[Non allineato]
\rightimplies ...\\
\leftimplies ...
\end{example}
\begin{codelatex}[Non allineato]
\begin{example}{}[Non allineato]
	\rightimplies ...\\
	\leftimplies ...
\end{example}
\end{codelatex}
\begin{example}{n}[Allineato]
	~{}\\
	\rightimplies ...\\
	\leftimplies ...
\end{example}
\begin{codelatex}[Allineato]
\begin{example}{n}[Allineato]
~{}\\
\rightimplies ...\\
\leftimplies ...
\end{example}
\end{codelatex}
\paragraph{\textbackslash rightinclude, \textbackslash leftinclude}
\begin{equation*}
	\rightinclude\qquad\qquad\leftinclude
\end{equation*}
\textbf{\textit{Descrizione:}} un'inclusione destra e sinistra con parentesi tonda e spazio incluso.\\
\textbf{\textit{Modalità d'uso:}} nessun argomento aggiuntivo.\\
\textbf{\textit{Scopo:}} per indicare quale inclusione di un'uguaglianza si sta dimostrando.
\begin{example}{}
	\rightinclude ...\\
	\leftinclude ...
\end{example}
\begin{codelatex}
\rightinclude ...\\
\leftinclude ...
\end{codelatex}
\paragraph{\textbackslash viff}
\begin{equation*}
	\viff
\end{equation*}
\textbf{\textit{Descrizione:}} una doppia implicazione \inlinecode{\textbackslash iff} $(\iff)$ verticale.\\
\textbf{\textit{Modalità d'uso:}} ambiente matematico, nessun argomento aggiuntivo.\\
\textbf{\textit{Scopo:}} per doppie implicazioni con enunciati su più linee.
\begin{example}{n}
	\begin{equation*}
		\begin{array}{c}
			...\\
			\viff\\
			...
		\end{array}
	\end{equation*}
\end{example}
\begin{codelatex}
\begin{equation*}
	\begin{array}{c}
		...\\
		\viff\\
		...
	\end{array}
\end{equation*}
\end{codelatex}
\section{Insiemi numerici e lettere}
\paragraph{\textbackslash N, \textbackslash Z, \textbackslash Q, \textbackslash R, \textbackslash Irr, \textbackslash C, \textbackslash Hb, \textbackslash Ob, \textbackslash Sb, \textbackslash K, \textbackslash F, \textbackslash kac, \textbackslash A}
\begin{equation*}
	\N\quad\Z\quad\Q\quad\R\quad\Irr\quad\C\quad\Hb\quad\Ob\quad\Sb\quad\K\quad\F\quad\kac\quad\A
\end{equation*}
\textbf{\textit{Descrizione:}} insiemi numerici e campi algebrici importanti.
\begin{multicols}{2}
	\begin{itemize}
		\item $\N$: numeri naturali;
		\item $\Z$: numeri interi;
		\item $\Q$: numeri razionali;
		\item $\R$: numeri reali;
		\item $\Irr$: numeri irrazionali;
		\item $\C$: numeri complessi;
		\item $\Hb$: quaternioni;
		\item $\Ob$: ottonioni;
		\item $\Sb$: sedenioni;
		\item $\K$: campo generico;
		\item $\F$: campo finito;
		\item $\kac$: campo algebricamente chiuso;
		\item $\A$: numero algebrico.
	\end{itemize}
\end{multicols}
\noindent\textbf{\textit{Modalità d'uso:}} ambiente matematico, nessun argomento aggiuntivo.
\begin{codelatex}
$\N\quad\Z\quad\Q\quad\R\quad\Irr\quad\C\quad\Hb\quad\Ob\quad\Sb\quad\K\quad\F\quad\kac\quad\A$
\end{codelatex}
\paragraph{\textbackslash epsilon, \textbackslash varepsilon}
\begin{equation*}
	\epsilon\qquad\varepsilon
\end{equation*}
\textbf{\textit{Descrizione:}} epsilon.\\
\textbf{\textit{Modalità d'uso:}} ambiente matematico, nessun argomento aggiuntivo.\\
\textbf{\textit{Osservazioni:}} scambia la definizione originale di \inlinecode{\textbackslash epsilon} con quella di \inlinecode{\textbackslash varepsilon}, dato che è un simbolo più ``\textit{carino}''.
\begin{codelatex}
$\epsilon\qquad\varepsilon$
\end{codelatex}
\paragraph{\textbackslash phi, \textbackslash varphi}
\begin{equation*}
	\phi\qquad\varphi
\end{equation*}
\textbf{\textit{Descrizione:}} phi.\\
\textbf{\textit{Modalità d'uso:}} ambiente matematico, nessun argomento aggiuntivo.\\
\textbf{\textit{Osservazioni:}} scambia la definizione originale di \inlinecode{\textbackslash phi} con quella di \inlinecode{\textbackslash varphi}, dato che è un simbolo più ``\textit{carino}''.
\begin{codelatex}
$\phi\qquad\varphi$
\end{codelatex}
\section{Logica}
\subsection{Teoria degli insiemi}
\paragraph{\textbackslash Set}
\begin{equation*}
	\Set{1,2,3,\ldots}\qquad\Set{x\in A | \ldots}
\end{equation*}
\textbf{\textit{Descrizione:}} insieme per elencazione o per caratteristica.\\
\textbf{\textit{Modalità d'uso:}} ambiente matematico;
\begin{itemize}
	\item \inlinecode{\#1} sono gli elementi;
	\item \inlinecode{\#2} \textit{(opzionale)} è la proprietà che caratterizza degli elementi.
\end{itemize}
\begin{codelatex}
$\Set{#1}$
$\Set{#1 | #2}$
\end{codelatex}
\begin{example}{}
	I primi sono $\Set{2,3,5,7,11,13,17,19,23,\ldots}$.\\
	Data una base $\Set{v_1,\ldots,v_n}$, un reticolo in $\R^n$ è $\displaystyle\Lambda=\Set{\sum_{i=1}^na_iv_i | a_i\in\Z}$.
\end{example}
\begin{codelatex}
I primi sono $\Set{2,3,5,7,11,13,17,19,23,\ldots}$.\\
Data una base $\Set{v_1,\ldots,v_n}$, un reticolo in $\R^n$ è $\displaystyle\Lambda=\Set{\sum_{i=1}^na_iv_i | a_i\in\Z}$.\end{codelatex}
\paragraph{\textbackslash emptyset, \textbackslash varemptyset}
\begin{equation*}
	\emptyset\qquad\qquad\varemptyset
\end{equation*}
\textbf{\textit{Descrizione:}} insieme vuoto.\\
\textbf{\textit{Modalità d'uso:}} ambiente matematico, nessun argomento aggiuntivo.\\
\textbf{\textit{Osservazioni:}} scambia la definizione originale di \inlinecode{\textbackslash emptyset} con quella di \inlinecode{\textbackslash varnothing}, dato che è un simbolo più tondo e ``\textit{carino}''; il vecchio simbolo per \inlinecode{\textbackslash emptyset} è diventato \inlinecode{\textbackslash varemptyset}. A seconda del font in uso i due simboli potrebbero essere praticamente identici.
\begin{codelatex}
	$\emptyset\qquad\qquad\varemptyset$
\end{codelatex}
\paragraph{\textbackslash powerset}
\begin{equation*}
	\powerset{X}
\end{equation*}
\textbf{\textit{Descrizione:}} insieme delle parti di un insieme $X$.\\
\textbf{\textit{Modalità d'uso:}} ambiente matematico;
\begin{itemize}
	\item \inlinecode{\#1} è l'insieme (interno alle parentesi) di cui si vuol scrivere l'insieme delle parti.
\end{itemize}
\begin{codelatex}
$\powerset{#1}$
\end{codelatex}
\begin{example}{}
	Se $X=\{x,y,z\}$, allora $\powerset{X}=\Set{\emptyset,\{x\},\{y\},\{z\},\{x,y\},\{x,z\},\{y,z\},\{x,y,z\}}$.
\end{example}
\begin{codelatex}
Se $X=\{x,y,z\}$, allora $\powerset{X}=\Set{\emptyset,\{x\},\{y\},\{z\},\{x,y\},\{x,z\},\{y,z\},\{x,y,z\}}$.
\end{codelatex}
\section{Algebra}
\subsection{Teoria delle categorie}
\paragraph{\textbackslash cat}
\begin{equation*}
	\cat
\end{equation*}
\textbf{\textit{Descrizione:}} categoria.\\
\noindent\textbf{\textit{Modalità d'uso:}} ambiente matematico, nessun argomento aggiuntivo.
\begin{codelatex}
$\cat$
\end{codelatex}
\paragraph{\textbackslash ob}
\begin{equation*}
	\ob
\end{equation*}
\textbf{\textit{Descrizione:}} oggetti.\\
\noindent\textbf{\textit{Modalità d'uso:}} ambiente matematico, nessun argomento aggiuntivo.
\begin{codelatex}
	$\ob$
\end{codelatex}
\paragraph{\textbackslash obj}
\begin{equation*}
	\obj{\cat}
\end{equation*}
\textbf{\textit{Descrizione:}} oggetti di una categoria $\cat$.\\
\textbf{\textit{Modalità d'uso:}} ambiente matematico;
\begin{itemize}
	\item \inlinecode{\#1} è la categoria a cui appartengono gli oggetti;
\end{itemize}
\begin{codelatex}
$\obj{#1}$
\end{codelatex}
\textbf{\textit{Osservazioni:}} si basa sull'operatore \inlinecode{\textbackslash ob} $(\ob)$.
\begin{example}{}
Gli oggetti $\obj{\underbfsf{Ab}}$ sono i gruppi abeliani.
\end{example}
\begin{codelatex}
Gli oggetti $\obj{\underbfsf{Ab}}$ sono i gruppi abeliani.
\end{codelatex}
\paragraph{\textbackslash homo}
\begin{equation*}
	\homo{X,Y}\qquad\homo[\cat]{X,Y}\qquad\homo{\cat}
\end{equation*}
\textbf{\textit{Descrizione:}} classe degli morfismi tra oggetti $X$ e $Y$ di una categoria $\cat$ - o classe di tutti i morfismi di una categoria $\cat$\\
\textbf{\textit{Modalità d'uso:}} ambiente matematico;
\begin{itemize}
	\item \inlinecode{\#1} \textit{(opzionale)} è la categoria a cui appartengono i morfismi;
	\item \inlinecode{\#2} è il dominio e codominio dei morfismi o la categoria a cui appartengono i morfismi;
\end{itemize}
\begin{codelatex}
$\homo[#1]{#2}$
\end{codelatex}
\textbf{\textit{Osservazioni:}} si basa sull'operatore \inlinecode{\textbackslash hom} $(\hom)$.
Si può utilizzare per insiemi di morfismi anche non esplicitamente categoriali, come gli omomorfismi tra gruppi.
\begin{example}{}
	I morfismi in $\homo[\cat]{X,Y}$ sono chiamati \textit{frecce} da $X$ a $Y$.
\end{example}
\begin{codelatex}
I morfismi in $\homo[\cat]{X,Y}$ sono chiamati anche \textit{frecce} da $X$ a $Y$.
\end{codelatex}
\subsection{Gruppi classici}
\paragraph{\textbackslash GL, \textbackslash SL, \textbackslash Or, \textbackslash SO, \textbackslash E, \textbackslash SE, \textbackslash U, \textbackslash SU, \textbackslash Sp, \textbackslash US}
\begin{equation*}
	\GL\quad\SL\quad\Or\quad\SO\quad\E\quad\SE\quad\U\quad\SU\quad\Sp\quad\US
\end{equation*}
\textbf{\textit{Descrizione:}} gruppi matriciali classici e loro quozienti.
\begin{multicols}{2}
	\begin{itemize}
		\item $\GL$: gruppo generale lineare;
		\item $\SL$: gruppo lineare speciale;
		\item $\Or$: gruppo ortogonale;
		\item $\SO$: gruppo ortogonale speciale;
		\item $\E$: gruppo euclideo;
		\item $\SE$: gruppo euclideo speciale;
		\item $\U$: gruppo unitario;
		\item $\SU$: gruppo unitario speciale;
		\item $\Sp$: gruppo simplettico;
		\item $\US$: gruppo simplettico compatto;
	\end{itemize}
\end{multicols}
%TODO: add projective generale linear group
\noindent\textbf{\textit{Modalità d'uso:}} ambiente matematico, nessun argomento aggiuntivo.
\begin{codelatex}
$\GL\quad\SL\quad\Or\quad\SO\quad\E\quad\SE\quad\U\quad\SU\quad\Sp\quad\US$
\end{codelatex}
\subsection{Funzioni}
\paragraph{\textbackslash funct}
\begin{gather*}
	\funct{}{X}{Y}\qquad\funct{}{X}{Y}[x][y]\qquad\funct{}[f]{X}{Y}\qquad
	\funct{}[f]{X}{Y}[x][y]\\
	\funct{s}{X}{Y}\qquad\funct{s}{X}{Y}[x][x]\qquad\funct{s}[f]{X}{Y}\qquad
	\funct{s}[f]{X}{Y}[x][x]\\
	\funct{i}{X}{Y}\qquad\funct{i}{X}{Y}[x][x]\qquad\funct{i}[\iota]{X}{Y}\qquad
	\funct{i}[\iota]{X}{Y}[x][x]
\end{gather*}
\textbf{\textit{Descrizione:}} funzione $f$ tra $X$ e $Y$ che manda $x$ in $y$.\\
\noindent\textbf{\textit{Modalità d'uso:}} ambiente matematico;
\begin{itemize}
	\item \inlinecode{\#1} è l'opzione sul \textit{tipo} di funzione:
	\begin{itemize}
		\item Nessuno (o qualunque altro simbolo eccetto i successivi): funzione generica.
		\item \inlinecode{s}: funzione suriettiva;
		\item \inlinecode{i}: inclusione;
	\end{itemize}
	\item \inlinecode{\#2} \textit{(opzionale)} è il nome della funzione;
	\item \inlinecode{\#3} è il dominio della funzione;
	\item \inlinecode{\#4} è il codominio della funzione;
	\item \inlinecode{\#5} \textit{(opzionale)} è l'elemento in input;
	\item \inlinecode{\#6} \textit{(obbligatorio se è presente} \inlinecode{\#5}\textit{, altrimenti è opzionale)} è l'elemento in output.
\end{itemize}
\textbf{\textit{Osservazioni:}} \inlinecode{\textbackslash funct} può essere utilizzata a sua volta negli argomenti \inlinecode{\#5} e \inlinecode{\#6}.\\ %TODO: evitare il nesting in #1, #2 e #3?
\begin{codelatex}
$\funct{#1}[#2]{#3}{#4}[#5][#6]$
\end{codelatex}
\begin{codelatex}
\begin{gather*}
	\funct{}{X}{Y}\qquad\funct{}{X}{Y}[x][y]\qquad\funct{}[f]{X}{Y}\qquad
	\funct{}[f]{X}{Y}[x][y]\\
	\funct{s}{X}{Y}\qquad\funct{s}{X}{Y}[x][x]\qquad\funct{s}[f]{X}{Y}\qquad
	\funct{s}[f]{X}{Y}[x][x]\\
	\funct{i}{X}{Y}\qquad\funct{i}{X}{Y}[x][x]\qquad\funct{i}[\iota]{X}{Y}\qquad
	\funct{i}[\iota]{X}{Y}[x][x]
\end{gather*}
\end{codelatex}
\section{Geometria}
\subsection{Algebra lineare}
\paragraph{\textbackslash rk}
\begin{equation*}
	\rk A
\end{equation*}
\textbf{\textit{Descrizione:}} rango di una matrice.\\
\noindent\textbf{\textit{Modalità d'uso:}} ambiente matematico, nessun argomento aggiuntivo.\\
\begin{codelatex}
$\rk$
\end{codelatex}
\paragraph{\textbackslash codim}
\begin{equation*}
	\codim A
\end{equation*}
\textbf{\textit{Descrizione:}} codimensione di uno spazio vettoriale.\\
\noindent\textbf{\textit{Modalità d'uso:}} ambiente matematico, nessun argomento aggiuntivo.
\begin{codelatex}
$\codim$
\end{codelatex}
\paragraph{\textbackslash img}
\begin{equation*}
\img A
\end{equation*}
\textbf{\textit{Descrizione:}} immagine di un operatore.\\
\noindent\textbf{\textit{Modalità d'uso:}} ambiente matematico, nessun argomento aggiuntivo.
\begin{codelatex}
$\img$
\end{codelatex}
\paragraph{\textbackslash trc}
\begin{equation*}
	\trc A
\end{equation*}
\textbf{\textit{Descrizione:}} traccia di una matrice.\\
\noindent\textbf{\textit{Modalità d'uso:}} ambiente matematico, nessun argomento aggiuntivo.
\begin{codelatex}
$\trc$
\end{codelatex}
\subsection{Geometria Euclidea}
\begin{equation*}
	AB \slantparallel CD
\end{equation*}
\textbf{\textit{Descrizione:}} due rette $AB$ e $CD$ parallele.\\
\noindent\textbf{\textit{Modalità d'uso:}} ambiente matematico, nessun argomento aggiuntivo.
\begin{codelatex}
$\parallel$
\end{codelatex}
\subsection{Topologia}
\paragraph{\textbackslash topo}
\begin{equation*}
	\topo
\end{equation*}
\textbf{\textit{Descrizione:}} topologia.\\
\noindent\textbf{\textit{Modalità d'uso:}} ambiente matematico, nessun argomento aggiuntivo.\\
\textbf{\textit{Osservazioni:}} è una $T$ calligrafica che forza l'utilizzo del font \inlinecode{Computer Modern} anche quando il font calligrafico è stato modificato, come in questo documento.
\begin{codelatex}
$\topo$
\end{codelatex}
\paragraph{\textbackslash eucl}
\begin{equation*}
	\eucl
\end{equation*}
\textbf{\textit{Descrizione:}} topologia euclidea.\\
\noindent\textbf{\textit{Modalità d'uso:}} ambiente matematico, nessun argomento aggiuntivo.
\begin{codelatex}
$\eucl$
\end{codelatex}
\paragraph{\textbackslash basis}
\begin{equation*}
	\basis
\end{equation*}
\textbf{\textit{Descrizione:}} base di una topologia.\\
\noindent\textbf{\textit{Modalità d'uso:}} ambiente matematico, nessun argomento aggiuntivo.
\begin{codelatex}
$\basis$
\end{codelatex}
\paragraph{\textbackslash interior}
\begin{equation*}
	\interior{A}
\end{equation*}
\textbf{\textit{Descrizione:}} interno di un insieme.\\
\textbf{\textit{Modalità d'uso:}} ambiente matematico;
\begin{itemize}
	\item \inlinecode{\#1} è l'insieme di cui si vuol scrivere l'interno.
\end{itemize}
\begin{codelatex}
$\interior{#1}$
\end{codelatex}
\begin{example}{}
	Se un insieme $A$ è aperto, $\interior{A}=A$.
\end{example}
\begin{codelatex}
Se un insieme $A$ è aperto, $\interior{A}=A$.
\end{codelatex}
\paragraph{\textbackslash unint}
\begin{equation*}
	\unint
\end{equation*}
\textbf{\textit{Descrizione:}} intervallo unitario.\\
\noindent\textbf{\textit{Modalità d'uso:}} ambiente matematico, nessun argomento aggiuntivo.
\begin{codelatex}
$\unint$
\end{codelatex}
\section{Topologia algebrica}
\paragraph{\textbackslash homotopy}
\begin{equation*}
	\homotopy{X,x_0}\qquad\homotopy[q]{X,x_0}\qquad\homotopy{X}\qquad\homotopy[q]{X}
\end{equation*}
\textbf{\textit{Descrizione:}} gruppo di omotopia (superiore) o gruppo fondamentale di uno spazio topologico $X$, eventualmente puntato in $x$.\\
\textbf{\textit{Modalità d'uso:}} ambiente matematico;
\begin{itemize}
	\item \inlinecode{\#1} \textit{(opzionale)} è l'ordine del gruppo di omotopia;
	\item \inlinecode{\#2} è lo spazio topologico (eventualmente puntato) di cui calcolare il gruppo di omotopia;
\end{itemize}
\begin{codelatex}
	$\homotopy[#1]{#2}$
\end{codelatex}
\begin{example}{}
	Il gruppo fondamentale della sfera è $\homotopy[1]{S^2,1}=0$.
\end{example}
\begin{codelatex}
Il gruppo fondamentale della sfera è $\homotopy[1]{S^2,1}=0$.
\end{codelatex}
\paragraph{\textbackslash homology}
\begin{equation*}
	\homology{X}\qquad\homology[q]{X}
\end{equation*}
\textbf{\textit{Descrizione:}} gruppo di omologia di uno spazio topologico $X$\\
\textbf{\textit{Modalità d'uso:}} ambiente matematico;
\begin{itemize}
	\item \inlinecode{\#1} \textit{(opzionale)} è l'ordine del gruppo di omologia;
	\item \inlinecode{\#2} è lo spazio topologico (o opportuno oggetto topologico) di cui calcolare il gruppo di omologia;
\end{itemize}
\begin{codelatex}
	$\homology[#1]{#2}$
\end{codelatex}
\begin{example}{}
	Il gruppo di omologia singolare della sfera $n$-dimensionale è $\homotopy[q]{S^n,1}=\Z$ se $q=0$ o $n$, altrimenti è $0$.
\end{example}
\begin{codelatex}
Il gruppo di omologia singolare della sfera $n$-dimensionale è $\homotopy[q]{S^n,1}=\Z$ se $q=0$ o $n$, altrimenti è $0$.
\end{codelatex}
\paragraph{\textbackslash deRham}
\begin{equation*}
	\deRham{M}\qquad\deRham[q]{M}
\end{equation*}
\textbf{\textit{Descrizione:}} gruppo di coomologia di De Rham di una varietà differenziabile $M$\\
\textbf{\textit{Modalità d'uso:}} ambiente matematico;
\begin{itemize}
	\item \inlinecode{\#1} \textit{(opzionale)} è l'ordine del gruppo di coomologia di De Rham;
	\item \inlinecode{\#2} è la varietà differenziabile di cui calcolare il gruppo di coomologia;
\end{itemize}
\begin{codelatex}
	$\deRham[#1]{#2}$
\end{codelatex}
\begin{example}{}
	Il gruppo di coomologia di De Rham della sfera $n$-dimensionale è $\deRham[q]{S^n,1}=\R$ se $q=0$ o $n$, altrimenti è $0$.
\end{example}
\begin{codelatex}
Il gruppo di coomologia di De Rham della sfera $n$-dimensionale è $\deRham[q]{S^n,1}=\R$ se $q=0$ o $n$, altrimenti è $0$.
\end{codelatex}
\section{Matematiche complementari} %TODO: don't have commands?
\section{Analisi matematica}
\paragraph{\textbackslash abs, \textbackslash norm}
\begin{equation*}
	\abs{\dots}\qquad\norm{\dots}
\end{equation*}
\textbf{\textit{Descrizione:}} valore assoluto (e norma) di un'espressione.\\
\textbf{\textit{Modalità d'uso:}} ambiente matematico;
\begin{itemize}
	\item \inlinecode{\#1} è l'espressione di cui si vuol prendere il valore assoluto (o norma).
\end{itemize}
\begin{codelatex}
$\abs{#1}$
$\norm{#1}$
\end{codelatex}
\textbf{\textit{Osservazioni:}} l'altezza delle parentesi cambia in base al contenuto di \inlinecode{\#1}.
\begin{example}{}
Un esempio di funzione non integrabile secondo Lebesgue è $\frac{\sin x}{x}$, in quanto
\begin{equation*}
	\int_{+\infty}^{+\infty}\abs{\frac{\sin x}{x}}dx=+\infty.
\end{equation*}
I vettori unitari hanno norma $\norm{\mathbf{x}}=1$.
\end{example}
\begin{codelatex}
Un esempio di funzione non integrabile secondo Lebesgue è $\nicefrac{\sin x}{x}$, in quanto
\begin{equation*}
	\int_{+\infty}^{+\infty}\abs{\frac{\sin x}{x}}dx=+\infty.
\end{equation*}
I vettori unitari hanno norma $\norm{\mathbf{x}}=1$.
\end{codelatex}
\paragraph{\textbackslash ceil, \textbackslash floor}
\begin{equation*}
	\floor{\dots}\qquad\ceil{\dots}
\end{equation*}
\textbf{\textit{Descrizione:}} funzione \textit{floor} (o parte intera) e funzione \textit{ceiling} di un'espressione.\\
\textbf{\textit{Modalità d'uso:}} ambiente matematico;
\begin{itemize}
	\item \inlinecode{\#1} è l'espressione di cui si vuol fare il floor/parte intera o ceiling.
\end{itemize}
\begin{codelatex}
	$\floor{#1}$
	$\ceil{\dots}$
\end{codelatex}
\textbf{\textit{Osservazioni:}} l'altezza delle parentesi cambia in base al contenuto di \inlinecode{\#1}.
\begin{example}{}
	L'operazione \textit{modulo} può essere espressa come $\displaystyle x\mod y=x-y\floor{\frac{x}{y}}$.\\
	Il numero di cifre in base $b$ di un intero positivo $k$ è $\floor{\log_b k}+1=\ceil{\log_b(k+1)}$.
\end{example}
\begin{codelatex}
L'operazione $\mod$ può essere espressa come $\displaystyle x\mod y=x-y\floor{\frac{x}{y}}$.\\
Il numero di cifre in base $b$ di un intero positivo $k$ è $\floor{\log_b k}+1=\ceil{\log_b(k+1)}$.
\end{codelatex}
\section{Probabilità e statistica matematica}
\section{Fisica e Fisica matematica}
\paragraph{\textbackslash fem}
\begin{equation*}
	\fem
\end{equation*}
\textbf{\textit{Descrizione:}} forza elettromotrice.\\
\noindent\textbf{\textit{Modalità d'uso:}} nessun argomento aggiuntivo.
\begin{codelatex}
\fem
\end{codelatex}
\paragraph{\textbackslash ddp}
\begin{equation*}
	\ddp
\end{equation*}
\textbf{\textit{Descrizione:}} differenza di potenziale.\\
\noindent\textbf{\textit{Modalità d'uso:}} nessun argomento aggiuntivo.
\begin{codelatex}
	\ddp
\end{codelatex}
\section{Analisi numerica}