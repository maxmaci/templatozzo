% SVN info for this file
\svnidlong
{$HeadURL$}
{$LastChangedDate$}
{$LastChangedRevision$}
{$LastChangedBy$}

\chapter*{Introduzione al Manualozzo\texttrademark\ }
\labelChapter{introduzione}
\thispagestyle{empty}
\begin{introduction}
‘‘Sai, per essere un matematico non aveva abbastanza immaginazione; ma ora è diventato un poeta e se la cava davvero bene.''
\begin{flushright}
	\textscsl{David Hilbert,} riferendosi \Ccancel[red]{a Marino Badiale} all'autore del Manualozzo\texttrademark\ .
\end{flushright}
\end{introduction}
\noindent Guardando la copertina di questo testo, dei potenziali lettori - sì, parlo con voi - si potrebbero chiedere: ‘‘Ma che diamine è un \textit{Manualozzo\texttrademark\ }?''
\vspace{3mm}
\lettrine[findent=1pt, nindent=0pt]{\textbf{M}}{\textbf{anualozzo\texttrademark\ }} s. m. [der. di \textit{manuale}, col suff. \textit{-ozzo}]. - Appunti di lezioni universitarie scritti da studenti, senza troppe pretese di formalità e potenzialmente non totalmente corretti, ma sono comunque meglio che niente.\footnote{Nota per l'ufficio legale: il \texttrademark\ in Manualozzo\texttrademark\ non è legalmente vincolante - per il momento.}
\vspace{3mm}\\

\newpage
\thispagestyle{empty}
~\vfill
\begin{center}
	Prima edizione, compilato il \today.\\
			\includegraphics[trim=0cm 0cm 0cm 0cm,clip,scale=0.5]{images/Cc-by-nc-sa_icon.pdf}\\
	{\footnotesize This work is licensed under a \href{https://creativecommons.org/licenses/by-sa/4.0/}{Attribution-NonCommercial-ShareAlike 4.0 International.}}
\end{center}
\newpage
%\section*{Note generali alla lettura}
%Quanto indicato con il simbolo ⋆ sono degli \textit{approfondimenti non necessari} - ma possono essere comunque utili ed interessanti per un lettore curioso.\\
%Le sezioni ‘‘Eserciziamoci!'' sono dedicate ad esercizi con corrispettive soluzioni: sono simili talvolta a dei risultati teorici già visti, ma tendono ad essere più applicativi.
%\section*{Note per gli environment}
%Se alcuni professori sono noti per abusare le notazioni, i Manualozzi\texttrademark\ sono noti per abusare di \textit{environment} - gli ambienti colorati che vedrete in queste pagine; di seguito ci sono alcune informazioni su di essi.
%
%\noindent\textit{Teoremi}, \textit{proposizioni}, \textit{lemmi} e \textit{corollari} possono essere seguiti da una \textit{dimostrazione}, come nell'esempio di seguito...
%\begin{theoremanote}[Esistenza dei monopoli magnetici]
%	Esistono i monopoli magnetici.
%\end{theoremanote}
%\begin{demonstration}
%	La dimostrazione si basa sulla congettura verificabile macroscopicamente che l'atomo sia indivisibile e indistruttibile. Supponiamo per assurdo che un magnete - ossia un dipolo magnetico - si possa dividere sempre in dipoli magnetici. Possiamo supporre senza particolari problemi di dividerlo fino ai suoi atomi fondamentali: poiché sono indivisibili e non sono dipoli, si ha un assurdo. È immediato quindi immaginare che gli atomi siano i monopoli magnetici dell'enunciato.
%\end{demonstration}
%... oppure essere forniti \textit{senza} dimostrazione e quindi nell'enunciato troverete alla fine il simbolo $\square$:
%\begin{corollarynote}[Validità della teoria delle stringhe]
%	Sulla base del teorema precedente per induzione sul numero di atomi esistono le stringhe.
%\end{corollarynote}
%\noindent Alcuni degli \textit{environment} più comuni dopo questi sono le \textit{osservazioni} e gli \textit{esempi}, che sono autoesplicativi. Ci sono anche altri \textit{environment}, meno comuni, fra cui...
%\begin{digression}
%	Sono argomenti \textit{non prettamente trattati} in questo corso che, tuttavia, hanno un legame con esso: possono \textit{aggiungere informazioni} e punti di vista a qualcosa visto nei corsi precedenti oppure fornire delle \textit{anticipazioni} per dei corsi futuri.
%\end{digression}
%\begin{attention}
%	Sono delle osservazioni mirate e rivolte spesso a segnalare \textit{errori} frequenti, dovuti principalmente a proprietà che \textit{non} si verificano in quel dato tangente. 
%\end{attention}
%\begin{intuit}
%	Sono delle interpretazioni \textit{euristiche} di una definizione difficile o di un risultato ostico che possono aiutare a capire il perché di tale cosa - per quanto non siano sempre valide a livello formale. 
%\end{intuit}
%\section*{Note per gli elenchi delle definizioni e dei teoremi}
%In fondo al Manualozzo si possono trovare degli elenchi con tutte le definizioni, assiomi e risultati teorici visti: ognuno di essi è indicato nel formato \textbf{\textsc{X\#.\#.\#. TITOLO}}, dove \textbf{\textsc{X}} è una \textit{sigla} per indicare il tipo di definizione/risultato, mentre \textbf{\textsc{\#.\#.\#.}} individua il \textit{capitolo}, la \textit{sezione} e il \textit{numero} per quell'oggetto nella sezione. I significati delle sigle sono elencati di seguito:
%\begin{multicols}{2}
%	\begin{itemize}
%		%\item \textbf{\textsc{A}}: Assioma.
%		\item \textbf{\textsc{D}}: Definizione.
%		\item \textbf{\textsc{T}}: Teorema.
%		\item \textbf{\textsc{PR}}: Proposizione.
%		\item \textbf{\textsc{L}}: Lemma.
%		\item \textbf{\textsc{C}}: Corollario.
%		\item \textbf{\textsc{PT}}: Proprietà.
%	\end{itemize}
%\end{multicols}


